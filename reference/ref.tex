\documentclass[11pt]{article}
\usepackage[fleqn]{amsmath}
\usepackage{amssymb, graphicx}
\usepackage{geometry}
\usepackage{longtable}
\usepackage{mathtools}
\usepackage{chngcntr}
\usepackage{authblk}
\usepackage{babel}
\usepackage{dsfont}
\usepackage{subcaption}
\usepackage{fancyhdr}
\usepackage{abstract}
\usepackage{epstopdf}
\usepackage{booktabs}
\usepackage{algorithm}
\usepackage{algorithmicx}
\usepackage{algpseudocode}
\usepackage{svg}
\usepackage{amsthm}
\usepackage{mathptmx}
\usepackage{xcolor}
\usepackage{array}
\usepackage{bbding}
\usepackage{multirow}
\usepackage{enumitem}
\usepackage{tikz}
\usetikzlibrary{positioning, arrows.meta, shapes.geometric}

\usepackage{hyperref}
\usepackage{cleveref}

\geometry{margin=1in}

\newtheorem{theorem}{Theorem}[section]
\newtheorem{definition}{Definition}[section]
\newtheorem{proposition}{Proposition}[section]
\newtheorem{remark}{Remark}[section]

\title{The \texorpdfstring{$\delta$}{delta}-Bounded \texorpdfstring{$\epsilon$}{epsilon}-Non-Dominated Multi-Objective User Equilibrium (\textbf{$\delta$-EBR-MUE}): Model, Theory and Approximation}
\author{Your Name}
\date{}

\begin{document}
\maketitle

\begin{abstract}
This paper introduces a novel multi-objective user equilibrium model under bounded rationality, termed the \emph{$\delta$-EBR-MUE}, which integrates two rationality constraints: (i) weakly $\epsilon$-non-dominated path choice, and (ii) travel time boundedness relative to the shortest path. We provide a complete mathematical formulation, establish existence via a set-valued fixed-point argument, and propose a support function approximation for the non-dominated frontier. Numerical experiments validate the behavioral implications and demonstrate the diversity of solutions beyond classical DUE or BRUE models.
\end{abstract}

\section{Notation and Preliminaries}

Let $\mathcal{W}$ denote the set of origin-destination (OD) pairs. For each OD pair $w \in \mathcal{W}$, let $\mathcal{P}^w$ denote the set of all feasible paths, and $d^w$ the total travel demand.

Let $\mathbf{f} = (f_p^w)_{p \in \mathcal{P}^w, w \in \mathcal{W}}$ denote the path flow vector. The feasible flow set is defined as:
\[
\Lambda := \left\{ \mathbf{f} \in \mathbb{R}_+^{|\mathcal{P}|} : \sum_{p \in \mathcal{P}^w} f_p^w = d^w, \ \forall w \in \mathcal{W} \right\}.
\]

Each path $p$ is associated with a bi-objective cost vector:
\[
\mathbf{C}_p(\mathbf{f}) := \left(t_p(\mathbf{f}),\ c_p(\mathbf{f}) \right),
\]
where $t_p(\mathbf{f})$ and $c_p(\mathbf{f})$ represent travel time and monetary cost under flow $\mathbf{f}$.

\subsection*{Dominance and $\epsilon$-Non-Dominance}

\begin{definition}[Strict Dominance]
Given two paths $p$ and $q$ for OD pair $w$, path $q$ \textbf{strictly dominates} path $p$, denoted
\[
\mathbf{C}_q(\mathbf{f}) \prec \mathbf{C}_p(\mathbf{f}),
\]
if and only if:
\[
t_q(\mathbf{f}) < t_p(\mathbf{f}) \quad \text{and} \quad c_q(\mathbf{f}) < c_p(\mathbf{f}).
\]
\end{definition}

\begin{definition}[$\boldsymbol{\epsilon}$-Strict Dominance]
For tolerance vector $\boldsymbol{\epsilon} = (\epsilon_t, \epsilon_m)$, path $q$ \textbf{$\epsilon$-strictly dominates} path $p$ if:
\[
t_q(\mathbf{f}) + \epsilon_t < t_p(\mathbf{f}) \quad \text{and} \quad c_q(\mathbf{f}) + \epsilon_m < c_p(\mathbf{f}).
\]
\end{definition}

\begin{definition}[$\boldsymbol{\epsilon}$-Weakly Non-Dominated Path]
Path $p$ is \textbf{$\epsilon$-weakly non-dominated} if it is not $\epsilon$-strictly dominated by any $q \in \mathcal{P}^w$, i.e.,
\[
\phi_p^{\epsilon}(\mathbf{f}) := \max_{q \in \mathcal{P}^w} \max \left\{ t_q - t_p + \epsilon_t,\ c_q - c_p + \epsilon_m \right\} \le 0.
\]
\end{definition}

\begin{figure}[ht]
    \centering
    % The image file 'epsilon_dominance_illustration.png' was not found during compilation.
    \includegraphics[width=0.6\textwidth]{../reference/fig/epsilon_dominance_illustration.png}
    \caption{Illustration of $\epsilon$-dominance in the time-cost objective space. The red box indicates the $\epsilon$-tolerance area around the reference path $p$. Paths inside this box are not $\epsilon$-strictly dominating $p$. Path $q_1$ strictly dominates $p$, $q_2$ $\epsilon$-dominates $p$, and $q_3$ does not dominate $p$.}
\end{figure}

\subsection*{Example: $\varepsilon$-Nondominated Flow Region Between Two Paths}

Consider two alternative paths \( p \) and \( q \) connecting the same origin-destination pair. Let the scalar flow on path \( p \) be denoted as \( f_p \), and on path \( q \) as \( f_q \). The bi-objective cost vectors for these two paths are given by:

\[
\begin{aligned}
\mathbf{C}_p(f) &= 
\begin{pmatrix}
C_p^1(f) \\
C_p^2(f)
\end{pmatrix}
= 
\begin{pmatrix}
2 + f_p \\
4 + f_p^2
\end{pmatrix}, \\
\mathbf{C}_q(f) &= 
\begin{pmatrix}
C_q^1(f) \\
C_q^2(f)
\end{pmatrix}
= 
\begin{pmatrix}
1 + 2f_q \\
5 + f_q
\end{pmatrix}.
\end{aligned}
\]

We consider an $\varepsilon$-dominance relation with tolerance vector 
\(\boldsymbol{\varepsilon} = (\varepsilon^1, \varepsilon^2) = (0.5, 0.5)\).
The cost vector \( \mathbf{C}_q(f) \) is said to $\varepsilon$-dominate \( \mathbf{C}_p(f) \) if:

\[
C_q^1(f) \leq C_p^1(f) - \varepsilon^1
\quad \text{and} \quad
C_q^2(f) \leq C_p^2(f) - \varepsilon^2.
\]

Therefore, the set of flow pairs \( (f_p, f_q) \in \mathbb{R}_+^2 \) under which path \( p \) is not $\varepsilon$-dominated by path \( q \) is:

\[
\mathcal{F}_{\varepsilon} = \left\{
(f_p, f_q) \in \mathbb{R}_+^2 \;\middle|\;
C_q^1(f) > C_p^1(f) - \varepsilon^1
\quad \text{or} \quad
C_q^2(f) > C_p^2(f) - \varepsilon^2
\right\}.
\]

This region can be visualised in the \( (f_p, f_q) \)-space as the set where the $\varepsilon$-dominance condition does not hold. Due to the nonlinear nature of the cost functions, the set \( \mathcal{F}_\varepsilon \) is generally non-convex.




\section{Mathematical Formulation of the \texorpdfstring{$\delta$}{delta}-EBR-MUE Model}

Let $\mathcal{W}$ be the set of origin-destination (OD) pairs, and $\mathcal{P}^w$ the set of feasible paths for $w \in \mathcal{W}$. For each path $p \in \mathcal{P}^w$:
\begin{itemize}
    \item $f_p^w$: flow on path $p$;
    \item $\mathbf{C}_p = (t_p(\mathbf{f}), c_p(\mathbf{f}))$: travel time and cost;
    \item $T_w^{\mathrm{UE}}(\mathbf{f}) := \min_{p \in \mathcal{P}^w} t_p(\mathbf{f})$.
\end{itemize}

A path $p$ is said to be $\epsilon$-non-dominated and $\delta$-bounded if:
\begin{align*}
    & \text{(i) } \nexists q \in \mathcal{P}^w: \ \mathbf{C}_q(\mathbf{f}) \prec \mathbf{C}_p(\mathbf{f}) - \boldsymbol{\epsilon}, \\
    & \text{(ii) } t_p(\mathbf{f}) \leq T_w^{\mathrm{UE}}(\mathbf{f}) + \delta.
\end{align*}

Let $\phi_p^{\epsilon}(\mathbf{f}) := \max_{q \in \mathcal{P}^w} \max\left\{t_q - t_p + \epsilon_t, c_q - c_p + \epsilon_m \right\}$ and $\phi_p^{\delta}(\mathbf{f}) := t_p(\mathbf{f}) - T_w^{\mathrm{UE}}(\mathbf{f}) - \delta$. The \textbf{$\delta$-EBR-MUE} conditions are:
\begin{align*}
    & f_p^w \cdot \phi_p^{\epsilon}(\mathbf{f}) = 0, \quad \phi_p^{\epsilon}(\mathbf{f}) \ge 0, \\
    & f_p^w \cdot \phi_p^{\delta}(\mathbf{f}) = 0, \quad \phi_p^{\delta}(\mathbf{f}) \le 0, \\
    & \sum_{p \in \mathcal{P}^w} f_p^w = d^w, \quad f_p^w \ge 0.
\end{align*}

\paragraph{Interpretation of $\phi_p^\epsilon(\mathbf{f})$.}
The expression $\phi_p^\epsilon(\mathbf{f}) := \max_{q \in \mathcal{P}^w} \max \left\{ t_q - t_p + \epsilon_t,\ c_q - c_p + \epsilon_m \right\}$ quantifies the worst-case $\epsilon$-dominance violation for path $p$.

\begin{itemize}
    \item The inner $\max$ evaluates whether any path $q$ offers a sufficient improvement in \emph{either} time or cost to make path $p$ strictly dominated beyond the tolerance $\epsilon$.
    \item The outer $\max$ scans all such potential dominating paths $q$.
\end{itemize}

If $\phi_p^\epsilon(\mathbf{f}) > 0$, it means there exists a path $q$ that $\epsilon$-dominates $p$ in at least one dimension. Therefore, path $p$ is considered \emph{infeasible} under the $\delta$-EBR-MUE behavioral rule. Conversely, $\phi_p^\epsilon(\mathbf{f}) = 0$ implies that $p$ is weakly $\epsilon$-non-dominated.

\section{Existence of \texorpdfstring{$\delta$}{delta}-EBR-MUE}

Let the feasible set be:
\[
\Lambda := \left\{ \mathbf{f} \in \mathbb{R}_+^{|\mathcal{P}|} : \sum_{p \in \mathcal{P}^w} f_p^w = d^w, \forall w \in \mathcal{W} \right\}.
\]
Define a set-valued map:
\[
\Phi(\mathbf{f}) := \left\{ \tilde{\mathbf{f}} \in \Lambda : \tilde{f}_p^w > 0 \Rightarrow p \in \mathcal{A}_w^{\epsilon,\delta}(\mathbf{f}) \right\},
\]
where $\mathcal{A}_w^{\epsilon,\delta}(\mathbf{f})$ is the set of all paths satisfying the two criteria above. Under continuity of $\mathbf{C}_p$, compactness of $\Lambda$, and closedness of $\Phi$, Kakutani’s fixed-point theorem ensures existence of an equilibrium $\mathbf{f}^* \in \Phi(\mathbf{f}^*)$.



\section{Existence of \texorpdfstring{$\delta$}{delta}-EBR-MUE}

We now establish the existence of a flow vector $\mathbf{f}^* \in \Lambda$ satisfying the conditions of the $\delta$-EBR-MUE model.

\begin{theorem}[Existence of $\delta$-EBR-MUE]
Suppose that:
\begin{enumerate}
    \item The feasible set of flows $\Lambda := \left\{ \mathbf{f} \in \mathbb{R}_+^{|\mathcal{P}|} : \sum_{p \in \mathcal{P}^w} f_p^w = d^w, \ \forall w \in \mathcal{W} \right\}$ is nonempty, compact, and convex;
    \item The path cost functions $t_p(\mathbf{f})$, $c_p(\mathbf{f})$ are continuous with respect to $\mathbf{f}$;
    \item The parameters $\epsilon_t, \epsilon_m, \delta \ge 0$ are fixed;
\end{enumerate}
Then there exists a flow vector $\mathbf{f}^* \in \Lambda$ satisfying the following conditions for all OD pairs $w \in \mathcal{W}$ and all $p \in \mathcal{P}^w$:
\begin{align*}
f_p^{w*} > 0 \Rightarrow 
\begin{cases}
\phi_p^{\epsilon}(\mathbf{f}^*) = 0, & \text{($\epsilon$-non-dominance)} \\\\
t_p(\mathbf{f}^*) \le T_w^{\mathrm{UE}}(\mathbf{f}^*) + \delta, & \text{(time-bounded)}
\end{cases}
\end{align*}
\end{theorem}

\begin{proof}
We reformulate the equilibrium as a fixed point of a set-valued map.

Define the feasible flow set:
\[
\Lambda = \left\{ \mathbf{f} \in \mathbb{R}_+^{|\mathcal{P}|} : \sum_{p \in \mathcal{P}^w} f_p^w = d^w, \forall w \in \mathcal{W} \right\},
\]
which is nonempty, convex, and compact.

Define for each OD pair $w$ the set of acceptable paths under flow $\mathbf{f}$:
\[
\mathcal{A}_w^{\epsilon,\delta}(\mathbf{f}) := \left\{ p \in \mathcal{P}^w \ : \ \phi_p^\epsilon(\mathbf{f}) = 0,\ t_p(\mathbf{f}) \le T_w^{\mathrm{UE}}(\mathbf{f}) + \delta \right\}.
\]

Then define the set-valued mapping:
\[
\Phi(\mathbf{f}) := \left\{ \tilde{\mathbf{f}} \in \Lambda \ : \ \tilde{f}_p^w > 0 \Rightarrow p \in \mathcal{A}_w^{\epsilon,\delta}(\mathbf{f}), \ \forall w \in \mathcal{W} \right\}.
\]

We claim that this map $\Phi : \Lambda \rightrightarrows \Lambda$ satisfies the conditions of Kakutani’s fixed-point theorem:

\begin{itemize}
    \item The domain $\Lambda$ is compact and convex (from assumption);
    \item For each $\mathbf{f}$, the image $\Phi(\mathbf{f})$ is convex: since selecting flows on acceptable paths preserves linearity and the feasible set is convex;
    \item The graph of $\Phi$ is closed: because both $\phi_p^\epsilon(\mathbf{f})$ and $t_p(\mathbf{f})$ are continuous in $\mathbf{f}$, the acceptance set $\mathcal{A}_w^{\epsilon,\delta}(\mathbf{f})$ varies upper hemicontinuously with $\mathbf{f}$.
\end{itemize}

By Kakutani’s theorem, there exists $ \mathbf{f}^* \in \Lambda$ such that $ \mathbf{f}^* \in \Phi(\mathbf{f}^*) $, i.e., a $\delta$-EBR-MUE exists.
\end{proof}

\section*{Algorithm: Iterative Solution for $\delta$-Bounded Rational Multi-Objective Non-Strictly Dominated User Equilibrium ($\delta$-EBR-MUE)}

\textbf{Input:}
\begin{itemize}
    \item OD demand set \(\{ d_w \}_{w \in W}\)
    \item Path sets \(\{ \mathcal{P}_w \}_{w \in W}\) for each OD pair
    \item Time cost tolerance parameter \(\delta > 0\)
    \item Path cost function \(\mathbf{C}_p(\mathbf{f})\), including time and other objectives
    \item Initial feasible path flow \(\mathbf{f}^{(0)}\) satisfying demand conservation
    \item Maximum number of iterations \(K_{\max}\) and convergence tolerance \(\varepsilon\)
\end{itemize}

\vspace{0.5em}
\textbf{Output:} Path flow vector \(\mathbf{f}^*\) satisfying the $\delta$-EBR-MUE equilibrium conditions.

\vspace{1em}
\textbf{Algorithm Steps:}
\begin{enumerate}[label=\arabic*.]
    \item \textbf{Initialization:} Set iteration counter \(k = 0\) and initialize path flow \(\mathbf{f}^{(0)}\).
    
    \item \textbf{Compute Path Costs:} For current flow \(\mathbf{f}^{(k)}\), compute multi-objective path costs:
    \[
        \mathbf{C}_p^{(k)} = \mathbf{C}_p(\mathbf{f}^{(k)}),
    \]
    extracting the time cost component:
    \[
        t_p^{(k)} = \text{time component of } \mathbf{C}_p^{(k)}.
    \]
    
    \item \textbf{Determine Minimum Time Cost per OD Pair:} For each OD pair \(w\), find the shortest travel time among all paths:
    \[
        t_w^{*(k)} = \min_{p \in \mathcal{P}_w} t_p^{(k)}.
    \]
    
    \item \textbf{Filter Feasible Paths:} Construct the feasible path set for each OD pair:
    \[
        \mathcal{P}_w^{(k)} = \left\{ p \in \mathcal{P}_w \mid t_p^{(k)} \leq t_w^{*(k)} + \delta, \quad p \text{ is non-strictly dominated} \right\}.
    \]
    Here, a path \(p\) is non-strictly dominated if there does not exist another path \(p'\) such that:
    \[
        \mathbf{C}_{p'}^{(k)} < \mathbf{C}_p^{(k)} \quad \text{(all objectives strictly less)}.
    \]
    
    \item \textbf{Flow Reassignment Subproblem:} Fixing the feasible path sets \(\{\mathcal{P}_w^{(k)}\}\), solve the flow assignment problem:
    \[
        \min_{\mathbf{f} \geq 0} \sum_{w \in W} \sum_{p \in \mathcal{P}_w^{(k)}} \int_0^{f_p} C_p(s) \, ds,
    \]
    subject to:
    \[
        \sum_{p \in \mathcal{P}_w^{(k)}} f_p = d_w, \quad \forall w \in W.
    \]
    Denote the solution by \(\mathbf{f}^{(k+1)}\).
    
    \item \textbf{Check Convergence:} If
    \[
        \| \mathbf{f}^{(k+1)} - \mathbf{f}^{(k)} \| \leq \varepsilon,
    \]
    terminate and set \(\mathbf{f}^* = \mathbf{f}^{(k+1)}\).
    
    \item \textbf{Iteration:} Otherwise increment \(k \leftarrow k + 1\). If \(k > K_{\max}\), terminate with current solution; else return to Step 2.
\end{enumerate}

\vspace{1em}
\textbf{Flowchart summary:}

\begin{center}
\begin{tikzpicture}[node distance=2cm, auto,>=Latex]
  \node (start) [rectangle, draw, rounded corners, text width=6cm, align=center] {Initialize flow \(\mathbf{f}^{(0)}\), set \(k=0\)};
  \node (cost) [rectangle, draw, rounded corners, below=of start, text width=6cm, align=center] {Compute path costs \(\mathbf{C}^{(k)}\), extract time costs \(t^{(k)}\)};
  \node (mincost) [rectangle, draw, rounded corners, below=of cost, text width=6cm, align=center] {Compute minimum time cost \(t_w^{*(k)}\) for each OD \(w\)};
  \node (filter) [rectangle, draw, rounded corners, below=of mincost, text width=6cm, align=center] {Filter feasible paths satisfying \\ time cost \(\le t_w^{*(k)} + \delta\) and non-strict dominance};
  \node (assign) [rectangle, draw, rounded corners, below=of filter, text width=6cm, align=center] {Solve flow assignment subproblem \\ to obtain \(\mathbf{f}^{(k+1)}\)};
  \node (check) [diamond, draw, aspect=2, below=of assign, yshift=-1cm, text width=6cm, align=center] {Convergence check: \(\|\mathbf{f}^{(k+1)} - \mathbf{f}^{(k)}\| \le \varepsilon\)?};
  \node (stop) [rectangle, draw, rounded corners, right=of check, xshift=4cm, text width=5cm, align=center] {Output \(\mathbf{f}^*\) and stop};
  
  \draw[->] (start) -- (cost);
  \draw[->] (cost) -- (mincost);
  \draw[->] (mincost) -- (filter);
  \draw[->] (filter) -- (assign);
  \draw[->] (assign) -- (check);
  \draw[->] (check) -- node[above] {Yes} (stop);
  \draw[->] (check.west) -- node[above] {No} ++(-4,0) |- (cost);
\end{tikzpicture}
\end{center}


%\begin{algorithm}[H]
%\caption{δ-EBR-MUE 多目标有界理性用户均衡迭代算法}
%\label{alg:delta_ebr_mue}
%\begin{algorithmic}[1]
%\Require OD需求 \(\{d_w\}\),路径集合 \(\{\mathcal{P}_w\}\),容忍度 \(\delta\),初始流量 \(\mathbf{f}^{(0)}\),最大迭代次数 \(K_{\max}\),收敛阈值 \(\varepsilon\)
%\Ensure 满足 δ-EBR-MUE 的路径流量 \(\mathbf{f}^*\)

%\State \(k \gets 0\)

%\Repeat
%    \State 计算路径成本 \(\mathbf{C}^{(k)} \gets \mathbf{C}(\mathbf{f}^{(k)})\)
%    \State 计算时间成本 \(t_p^{(k)}\) 并求每个OD的最短时间成本:
%    \[
%        t_w^{*(k)} \gets \min_{p \in \mathcal{P}_w} t_p^{(k)}, \quad \forall w
%    \]
%    \State 构造满足条件的路径集合:
%    \[
%        \mathcal{P}_w^{(k)} \gets \{ p \in \mathcal{P}_w \mid t_p^{(k)} \le t_w^{*(k)} + \delta, \quad p \text{ 非严格支配} \}
%    \]
%    \State 在路径集合 \(\{\mathcal{P}_w^{(k)}\}\) 上,解流量分配子问题,更新流量:
%    \[
%        \mathbf{f}^{(k+1)} \gets \arg\min_{\mathbf{f} \ge 0} \sum_{w} \sum_{p \in \mathcal{P}_w^{(k)}} \int_0^{f_p} C_p(s) ds
%    \]
%    \[
%        \text{s.t. } \sum_{p \in \mathcal{P}_w^{(k)}} f_p = d_w, \quad \forall w
%    \]
%    \State \(k \gets k + 1\)
%\Until{ \(\|\mathbf{f}^{(k)} - \mathbf{f}^{(k-1)}\| \le \varepsilon\) or \(k > K_{\max}\)}

%\State \Return \(\mathbf{f}^* = \mathbf{f}^{(k)}\)
%\end{algorithmic}
%\end{algorithm}

\begin{algorithm}[H]
\caption{VI-based Algorithm for Solving $\delta$-Bounded $\boldsymbol{\epsilon}$-Non-strictly Dominated User Equilibrium ($\delta$-EBR-MUE)}
\label{alg:vi-delta-eps}
\begin{algorithmic}[1]
\Require Initial flow \(\mathbf{f}^{(0)}\), OD demands \(\{d^w\}\), path sets \(\{\mathcal{P}_w\}\), tolerance \(\delta > 0\), \(\boldsymbol{\epsilon} \in \mathbb{R}_+^m\), maximum iteration \(K_{\max}\)
\Ensure Equilibrium flow \(\mathbf{f}^*\)
\State Set \(k \gets 0\)
\Repeat
    \State Compute the cost vector \(\mathbf{C}_p(\mathbf{f}^{(k)}) = (C_p^1, \dots, C_p^m)\) for all \(p\)
    \For{each OD pair \(w \in \mathcal{W}\)}
        \State Find minimum time cost: \(\mathcal{T}_w = \min_{p \in \mathcal{P}_w} C_p^1(\mathbf{f}^{(k)})\)
        \State Define feasible path set:
        \[
        \hat{\mathcal{P}}_w := \left\{ p \in \mathcal{P}_w : C^1_p(\mathbf{f}^{(k)}) \leq \mathcal{T}_w + \delta,~\nexists~ q \in \mathcal{P}_w \text{ s.t. } \mathbf{C}_q \leq \mathbf{C}_p - \boldsymbol{\epsilon} \right\}
        \]
    \EndFor

    \State Define feasible flow set:
    \[
    \mathcal{F}^{\delta,\boldsymbol{\epsilon}} := \left\{ \mathbf{f} \geq 0 ~\middle|~ \sum_{p \in \mathcal{P}_w} f_p = d^w,~ f_p > 0 \Rightarrow p \in \hat{\mathcal{P}}_w,~\forall w \right\}
    \]

    \State Define operator \( F(\mathbf{f}) = \left( \mathbf{C}_p(\mathbf{f}) \right)_{p \in \cup_w \mathcal{P}_w} \)
    
    \State Solve the following variational inequality (projected method or proximal point method):
    \[
    \text{Find } \mathbf{f}^{(k+1)} \in \mathcal{F}^{\delta,\boldsymbol{\epsilon}} \text{ such that } \langle F(\mathbf{f}^{(k+1)}), \mathbf{f} - \mathbf{f}^{(k+1)} \rangle \geq 0, \quad \forall \mathbf{f} \in \mathcal{F}^{\delta,\boldsymbol{\epsilon}}
    \]
    
    \State \(k \gets k + 1\)
\Until{\(\|\mathbf{f}^{(k)} - \mathbf{f}^{(k-1)}\| \leq \text{tol}\) or \(k \geq K_{\max}\)}

\State \Return \(\mathbf{f}^{(k)}\)
\end{algorithmic}
\end{algorithm}




\begin{algorithm}[H]
\caption{$\delta$-bounded $\boldsymbol{\epsilon}$-Non-dominated User Equilibrium Solver}
\label{alg:delta-eps-ebr-mue}
\begin{algorithmic}[1]
\Require Initial flow vector \(\mathbf{f}^0\), OD demands \(\{d^w\}_{w \in \mathcal{W}}\), path sets \(\{\mathcal{P}_w\}\), tolerance parameters \(\delta > 0\), \(\boldsymbol{\epsilon} \in \mathbb{R}^m_+\), maximum iterations \(K_{\max}\)
\Ensure Approximate equilibrium flow \(\mathbf{f}^\ast\)
\State Set \(k \gets 0\)
\State Initialize \(\mathbf{f}^{(0)}\)
\Repeat
    \State Compute cost vectors \(\mathbf{C}_p(\mathbf{f}^{(k)})\) for all paths \(p\)
    \For{each OD pair \(w \in \mathcal{W}\)}
        \State Let \(\mathcal{T}_w^{(k)} = \min_{p \in \mathcal{P}_w} C^1_p(\mathbf{f}^{(k)})\) \Comment{Minimum travel time cost}
        \State Let \(\hat{\mathcal{P}}_w^{(k)} \gets \emptyset\)
        \For{each path \(p \in \mathcal{P}_w\)}
            \If{\(C^1_p(\mathbf{f}^{(k)}) \leq \mathcal{T}_w^{(k)} + \delta\) \textbf{and} not strictly \(\boldsymbol{\epsilon}\)-dominated by other paths in \(\mathcal{P}_w\)}
                \State Add \(p\) to \(\hat{\mathcal{P}}_w^{(k)}\)
            \EndIf
        \EndFor
    \EndFor
    \State Assign flow uniformly over \(\hat{\mathcal{P}}_w^{(k)}\) for each \(w\), respecting demand \(d^w\), to obtain \(\mathbf{f}^{(k+1)}\)
    \State \(k \gets k + 1\)
\Until{\(\|\mathbf{f}^{(k)} - \mathbf{f}^{(k-1)}\| \leq \text{tol}\) or \(k \geq K_{\max}\)}
\State \textbf{return} \(\mathbf{f}^{(k)}\)
\end{algorithmic}
\end{algorithm}




\section{Support Function Approximation of the \texorpdfstring{$\epsilon$}{epsilon}-Nondominated Set}

Let $\mathcal{C}^w := \{ \mathbf{C}_p : p \in \mathcal{P}^w \}$ and choose $\mathcal{U}_N \subset \mathbb{S}^1$ directions. Define:
\[
\rho_{\mathcal{C}^w}(\mathbf{u}) := \min_{p \in \mathcal{P}^w} \mathbf{u}^\top \mathbf{C}_p.
\]
Then:
\[
\mathcal{E}_\epsilon^w := \left\{ \mathbf{C}_p : \forall \mathbf{u} \in \mathcal{U}_N,\ \mathbf{u}^\top \mathbf{C}_p \le \rho_{\mathcal{C}^w}(\mathbf{u}) + \epsilon_\mathbf{u} \right\}.
\]
As $N \to \infty$, $\mathcal{E}_\epsilon^w$ converges in Hausdorff distance to the true weakly nondominated frontier.

\section{Limiting Case: $\boldsymbol{\epsilon} = 0$}

We consider the limiting case where the dominance tolerance $\boldsymbol{\epsilon} = (0, 0)$. In this case, the $\delta$-EBR-MUE model reduces to a sharper form, where users only accept paths that are \emph{not strictly dominated} in the cost vector space, while still satisfying the travel time bound.

\begin{definition}[{$\delta$-Bounded Weakly Non-Dominated User Equilibrium (\textbf{$\delta$-WNUE})}]
A feasible path flow $\mathbf{f}^*$ is said to satisfy the $\delta$-WNUE condition if:
\begin{align*}
f_p^w > 0 \ \Rightarrow \ 
\begin{cases}
\mathbf{C}_q(\mathbf{f}^*) \not\prec \mathbf{C}_p(\mathbf{f}^*), & \forall q \in \mathcal{P}^w \\[1ex]
t_p(\mathbf{f}^*) \le T_w^{\mathrm{UE}}(\mathbf{f}^*) + \delta, &
\end{cases}
\end{align*}
\end{definition}

\begin{proposition}
Every $\delta$-EBR-MUE with $\boldsymbol{\epsilon} = 0$ is a $\delta$-WNUE. Conversely, every $\delta$-WNUE is a $\delta$-EBR-MUE with $\boldsymbol{\epsilon} = 0$.
\end{proposition}

\begin{proof}
Immediate from the definitions: with $\boldsymbol{\epsilon} = 0$, the $\epsilon$-non-dominance condition becomes standard weak non-domination. The time-bound condition remains unchanged.
\end{proof}

\begin{remark}
As $\epsilon \to 0$, the $\delta$-EBR-MUE set converges to the set of $\delta$-WNUE flows. However, for any $\epsilon > 0$, the equilibrium permits a richer variety of solutions, possibly including near-dominated paths, thereby reflecting behavioural flexibility and tolerance.
\end{remark}


\section{Numerical Example}

Consider a simple network with three parallel paths. Let the travel time on each path be $t_p(f) = a_p + b_p f_p$ and cost $c_p$ be fixed. Set:
\begin{itemize}
    \item $a = [3, 5, 8]$, $b = [0.01, 0.01, 0.02]$;
    \item $c = [2, 1.5, 0.5]$ (monetary units);
    \item demand $d = 1$, $\epsilon = (1.5, 0.5)$, $\delta = 2$.
\end{itemize}

Compute:
\begin{enumerate}
    \item User Equilibrium (UE);
    \item Bounded Rational UE (BRUE);
    \item $\delta$-EBR-MUE using fixed-point iteration.
\end{enumerate}

Compare results in terms of used paths, average cost, and Pareto dominance.

\end{document}